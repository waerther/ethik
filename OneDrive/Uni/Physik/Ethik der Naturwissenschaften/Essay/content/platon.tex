\section*{Platons Idealismus und Gesprächsführung} 

Platon sieht trotz Sokrates' Skeptizismus eine objektive Wahrheit, \textit{das Gute}, als durchaus möglich an und stellt
dieses als höchstes Gut seiner Ideenhierarchie, bei der er strikt zwischen der für uns währnehmbaren Welt 
und einer Welt der Ideen und Formen unterscheidet. 
Nach Platon gebe die Sinneswahrnehmung keine echten Wahrheiten, da sie ständig veränderlich und trügerisch sei.
Die Quelle der Wahrheit sei der angeborene Geist, der das Wissen um die Wahrheit von Natur aus besitzt.
Die rechten Begriffe, den Ideen, seien reine Gedanken von dem, was Wirklichkeit erscheint.
Darin sind neben geometrischen idealisierten Formen, wie etwa Kreisen, Kugeln oder gleichseitigen (platonischen) Körpern 
auch höhere Konzepte von Moral und Ethik enthalten.

Zur Schaffung eines einhaltlichen Verständnisses dieser Ideen sieht Platon die Klärung der Grunfegriffe als unabdingbar.
In seinem in Dialogform verfassten Werk \enquote{Gorgias} \cite{gorgias} führt Platon den Sokratischen Dialog ein, 
um den semantischen Begriff der \textit{Rhetorik} zu erörtern. 
Dabei bemüht sich der im Werk sprechende Sokrates, durch kritisches Fragen eine Erkenntnis durch folgene Mittel zu gewinnen:

\begin{itemize}
    \item   Annahme, dass eine richtige Definition des Untersuchungsgegenstand existiert.
            Dabei sind Begriffserörterung durch Beispiele notwendig.
    \item   Sokratische Elenktik, das Aufdecken von Widersprüchen und deren Widerlegung. 
            Antworten hieraus bilden Prämissen für die nächste Frage.
    \item   Prüfen der Richtigkeit des Gegenstands Gesamtkonzept, Rückführung auf Werte und Tugenden. 
            Sind die Konsequenzen dieser Prüfung mit den vorangegangen Prämissen vereinbar?
\end{itemize}


\section*{Platons Kardinaltugenden und Staatslehre}

Im Werk \enquote{Der Staat} \cite{politeia} verfasst Platon erneut eine dialogische Definition von \textit{Gerechtigkeit}
und postuliert somit eine mögliche Verwirklichung eines idealen Staates.
Die Gesprächsteilnehmer kommen zu dem Schluss, dass Erfolg unweigerlich Zusammenarbeit erfordert.
Ungerechtes Handeln, das heißt Handeln entgegen dem Willen eines der Teilnehmer, untergräbt einen möglichen Erfolg.
Gerechtigkeit benötigt nach dieser Schlussfolgerung eine moralische Basis für ein Miteinander.

Des Weiteren stellen die Handelnden die Frage nach dem Zweck der Gerechtigkeit und erkennen, 
dass sie als eine pragmatische Lösung fungiert, welche durch die Gesetzgebung im sozialen Interesse umgesetzt wird.
Hierbei wird festgestellt, dass Gerechtigkeit weniger eine persönliche Aufgabe ist, als ein Mittel, 
eigene Interessen im akzeptieretem Rahmen umzusetzen und nicht um ihrer intrinsisch guten Selbstwillen da ist.
Platon führt zur Lösung dieses Missstands den Ausbau einer Staatslehre nach Vorbild von Tugenden und ethischem Handeln ein,
indem er die die Seele in zwei Grundkräfte teilt.

Platons Gleichnis des Seelenwagens stellt die Grundkräfte als ein gutes (Mut) und ein schlechtes (Begierde) Pferd dar, 
welche durch den Wagenlenker, der Weisheit und Grundkraft der Vernunft, auf das \textit{wahre Gute} gelenkt werden.
Die drei daraus folgenden Tugenden Vernunft, Mut und Begehren spiegeln sich schließlich nicht nur ethischen Handeln,
sondern auch im Verfassungsmodell wieder:
Den politischen Führern (Vernunft) sind dabei die Wächter (Mut) untegeordnet, welche das erwerbstätige Volk (Begehren) mäßigen.
Die Ausbildung aller drei Tugenden, so auch aller politischer Ämter, 
resultiert demnach in der vierten Kardinaltugend: Gerechtigkeit.


