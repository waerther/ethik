\section*{Einleitung}

Der antike griechische Philosoph Sokrates ist verbreitet bekannt als der Vater der abendländischen Philosophie.
Seine Skepsis gegenüber allem bestehenden Wissen und das systematische Hinterfragen führten ihn zu der Einsicht,
dass die vollkomme Erkenntnis von Wahrheit in Frage zu stellen oder prinzipiell auszuschließen ist.
Obwohl kein Werk ihm direkt zugesprochen werden kann, wird seine kritische und selbsthinterfragende Denkweise
von den prägenden Philosophen Platon und Aristoteles weitergeführt und ergänzt. 
Das Erreichen von unwiderlegbarem Wissen wird bei beiden Denkern einem guten Leben gegenübergestellt und damit 
das ethische Handeln mit der Erkenntnistheorie verknüpft.
