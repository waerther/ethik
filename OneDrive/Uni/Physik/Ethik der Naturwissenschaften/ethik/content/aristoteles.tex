\section*{Aristoteles' Tugendlehre und Eudaimonie}

Als Platons Schüler und geistiger Nachfolger griff Aristoteles die Teilung der Seele erneut auf,
unterteilte sie jedoch in einen vernunftbegabten und einen vernunftlosen Teil.
Ersterer wohnt auszuschließlich dem Menschen inne, während letzterer den vegetativen Teil aller Lebewesen darstellt,
wie etwa der Herzschlag oder das Empfinfungsvermögen.
Aristoteles hingegen fasst in seinem Werk \cite{nikethik} die Tugenden als Leistungen der Seele weiter und unterteilt
diese in durch Triebe bestimmte, \textit{ethische Tugenden} und den der Vernunft entspringenden, \textit{dianoёtischen Tugenden}.
Das ethische Handeln ergebe sich nach seiner \textit{Mesoteslehre} aus der gleichgewichteten Wahl der ethischen Tugenden
im Einklang mit der Verstandestugend der Einsicht. 

Im Gegensatz zu seinem Lehrer, vertritt Aristoteles nicht die platonische Idee als höchstes Ziel.
Nach ihm sind die Ziele hierarchisch angeordnet und endlich. 
Dash heißt, dass ein untergeordnetes Ziel zu willen eines höheren angestrebt wird.
Das Endziel hingegen muss nur um ihrer Selbstwillen erreicht werden, da andernfalls die Handlung zwecklos sei.
Das höchste Gut und Endziel der nikomachischen Ethik ist die Eudaimonie (Glückseligkeit), 
was aber nur durch bewusstes Handeln verwirklicht werden kann.

\section*{Aristoteles' Erkenntnistheorie}

Aristoteles wendet sich ebenfalls von Platons Überzeugung ab, die Wirklichkeit auszuschließlich in Ideen und Gedanken zu erkennen.
Während Platon prinzipiell Empirie als Quelle für Wissen ablehnt, sind nach der aristotelischen Lehre 
Schlüsse vom Allgemeinen auf das Besondere und umgekehrt durchaus möglich. 
So folgert bei einer Deduktion (Syllogismus) aus zwei allgemeinen Prämissen eine Konklusion auf den Einzelfall.
Analag geschieht die Induktion durch Erkennen einens Musters von vielen Einzelfällen eine Prjektion auf einen allgemeinen Gegenstand.
Damit setzt er den Grundstein der Logik und erdet die Theorie mit empirische Vorgänge.
Wahrheitsaussagen (Urteile) und deren Verknüpfungen bilden schließlich Beweise und schaffen somit eine Kategorisierung der Wirlichkeit.




